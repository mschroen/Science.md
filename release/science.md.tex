\documentclass[A4paper,]{article}
\usepackage{lmodern}
\usepackage{amssymb,amsmath}
\usepackage{ifxetex,ifluatex}
\usepackage{fixltx2e} % provides \textsubscript
\ifnum 0\ifxetex 1\fi\ifluatex 1\fi=0 % if pdftex
  \usepackage[T1]{fontenc}
  \usepackage[utf8]{inputenc}
\else % if luatex or xelatex
  \ifxetex
    \usepackage{mathspec}
  \else
    \usepackage{fontspec}
  \fi
  \defaultfontfeatures{Ligatures=TeX,Scale=MatchLowercase}
\fi
% use upquote if available, for straight quotes in verbatim environments
\IfFileExists{upquote.sty}{\usepackage{upquote}}{}
% use microtype if available
\IfFileExists{microtype.sty}{%
\usepackage{microtype}
\UseMicrotypeSet[protrusion]{basicmath} % disable protrusion for tt fonts
}{}
\usepackage[margin=1in]{geometry}
\usepackage[unicode=true]{hyperref}
\PassOptionsToPackage{usenames,dvipsnames}{color} % color is loaded by hyperref
\hypersetup{
            pdftitle={A Review of Gravitational Theories},
            colorlinks=true,
            linkcolor=Maroon,
            citecolor=Blue,
            urlcolor=Blue,
            breaklinks=true}
\urlstyle{same}  % don't use monospace font for urls
\usepackage{graphicx,grffile}
\makeatletter
\def\maxwidth{\ifdim\Gin@nat@width>\linewidth\linewidth\else\Gin@nat@width\fi}
\def\maxheight{\ifdim\Gin@nat@height>\textheight\textheight\else\Gin@nat@height\fi}
\makeatother
% Scale images if necessary, so that they will not overflow the page
% margins by default, and it is still possible to overwrite the defaults
% using explicit options in \includegraphics[width, height, ...]{}
\setkeys{Gin}{width=\maxwidth,height=\maxheight,keepaspectratio}
\IfFileExists{parskip.sty}{%
\usepackage{parskip}
}{% else
\setlength{\parindent}{0pt}
\setlength{\parskip}{6pt plus 2pt minus 1pt}
}
\setlength{\emergencystretch}{3em}  % prevent overfull lines
\providecommand{\tightlist}{%
  \setlength{\itemsep}{0pt}\setlength{\parskip}{0pt}}
\setcounter{secnumdepth}{5}
% Redefines (sub)paragraphs to behave more like sections
\ifx\paragraph\undefined\else
\let\oldparagraph\paragraph
\renewcommand{\paragraph}[1]{\oldparagraph{#1}\mbox{}}
\fi
\ifx\subparagraph\undefined\else
\let\oldsubparagraph\subparagraph
\renewcommand{\subparagraph}[1]{\oldsubparagraph{#1}\mbox{}}
\fi

% set default figure placement to htbp
\makeatletter
\def\fps@figure{htbp}
\makeatother

%%% MARTIN
    \usepackage{fancyhdr}
    \pagestyle{fancy}
    \fancyhead{}
    \fancyhead[RO,LE]{GRAVITATIONAL THEORIES}
    \fancyhead[LO,RE]{Einstein et al.}
    \fancyfoot{}
    \fancyfoot[LE,RO]{\thepage}
    
    \usepackage[version=4]{mhchem} % easy chem formulae with ce{H2O}
        \def\AE#1\par{\setlength{\fboxsep}{1pt}\colorbox{cyan}{\textcolor{white}{\texttt{AE}}} \color{cyan} #1\color{black}}
        \def\IN#1\par{\setlength{\fboxsep}{1pt}\colorbox{magenta}{\textcolor{white}{\texttt{IN}}} \color{magenta} #1\color{black}}
        \newcommand{\TODO}{\setlength{\fboxsep}{1pt}\colorbox{black!50}{\textcolor{white}{TODO}} }
%%%

\usepackage{subfig}
\AtBeginDocument{%
\renewcommand*\figurename{Figure}
\renewcommand*\tablename{Table}
}
\AtBeginDocument{%
\renewcommand*\listfigurename{List of Figures}
\renewcommand*\listtablename{List of Tables}
}
\usepackage{float}
\floatstyle{ruled}
\makeatletter
\@ifundefined{c@chapter}{\newfloat{codelisting}{h}{lop}}{\newfloat{codelisting}{h}{lop}[chapter]}
\makeatother
\floatname{codelisting}{Listing}
\newcommand*\listoflistings{\listof{codelisting}{List of Listings}}

\title{A Review of Gravitational Theories}
%%% MARTIN
    \usepackage{authblk}

          \author[1,C]{A. Einstein}
          \author[2]{I. Newton}
      

             \affil[1]{Federal Office for Intellectual Property, Bern, Switzerland}
          \affil[2]{University of Cambridge}
          \affil[C]{Correspondence: albert.einstein@gmail.com}
      
    \setcounter{Maxaffil}{0}
    \renewcommand\Affilfont{\itshape\small}
%%%
\date{}

\begin{document}
\maketitle
\begin{abstract}
The classical theory of gravitation has been revised to find a new relativistic theory of gravitation. Impact for society will be tremendous.
%%% MARTIN

    \vspace*{1em}
    Keywords:  Classical mechanics,  Relativistic mechanics, 
%%%
\end{abstract}

\section{Introduction}\label{sec:intro}

Recently, the theory of classical mechanics has been presented by Newton (\protect\hyperlink{ref-Newton1730}{1730}).

Lorem ipsum dolor sit amet, consetetur sadipscing elitr, sed diam nonumy eirmod tempor invidunt ut labore et dolore magna aliquyam erat, sed diam voluptua. At vero eos et accusam et justo duo dolores et ea rebum. Stet clita kasd gubergren, no sea takimata sanctus est Lorem ipsum dolor sit amet. Lorem ipsum dolor sit amet, consetetur sadipscing elitr, sed diam nonumy eirmod tempor invidunt ut labore et dolore magna aliquyam erat, sed diam voluptua. At vero eos et accusam et justo duo dolores et ea rebum. Stet clita kasd gubergren, no sea takimata sanctus est Lorem ipsum dolor sit amet.

\section{Material and Methods}\label{material-and-methods}

We make use of the method of \emph{intuition} to invent another theory (see Einstein \protect\hyperlink{ref-Einstein1905}{1905} and references therein). Occassionally, formulas were used, too (see e.g., eq.~\ref{eq:emc2}).

\section{Results and Discussion}\label{results-and-discussion}

The relativistic theory works much better than the classical theory, \(F=m\cdot a​\), (compare section~\ref{sec:intro}). Most probably because more complex equations are involved, like \(E=\gamma m_0c^2​\).
In Fig.~\ref{fig:theories} some concepts are shown that might or might not our findings.

\begin{figure}
\centering
\includegraphics{../fig/theories.png}
\caption{Some theories. Credit: Wikipedia.}\label{fig:theories}
\end{figure}

Lorem ipsum dolor sit amet, consetetur sadipscing elitr, sed diam nonumy eirmod tempor invidunt ut labore et dolore magna aliquyam erat, sed diam voluptua. At vero eos et accusam et justo duo dolores et ea rebum. Stet clita kasd gubergren, no sea takimata sanctus est Lorem ipsum dolor sit amet. Lorem ipsum dolor sit amet, consetetur sadipscing elitr, sed diam nonumy eirmod tempor invidunt ut labore et dolore magna aliquyam erat, sed diam voluptua. At vero eos et accusam et justo duo dolores et ea rebum. Stet clita kasd gubergren, no sea takimata sanctus est Lorem ipsum dolor sit amet.

\section{Conclusion and Outlook}\label{conclusion-and-outlook}

Relativistic mechanics is probably the best way to describe a new theory of gravitation.
The future will show whether there is any application of our theories.

Lorem ipsum dolor sit amet, consetetur sadipscing elitr, sed diam nonumy eirmod tempor invidunt ut labore et dolore magna aliquyam erat, sed diam voluptua. At vero eos et accusam et justo duo dolores et ea rebum. Stet clita kasd gubergren, no sea takimata sanctus est Lorem ipsum dolor sit amet. Lorem ipsum dolor sit amet, consetetur sadipscing elitr, sed diam nonumy eirmod tempor invidunt ut labore et dolore magna aliquyam erat, sed diam voluptua. At vero eos et accusam et justo duo dolores et ea rebum. Stet clita kasd gubergren, no sea takimata sanctus est Lorem ipsum dolor sit amet.
\appendix

\section{Some maths}\label{some-maths}

\begin{equation} E = m\cdot c^2\,, \label{eq:emc2}\end{equation}

Because people love to see equations.

\section*{References}\label{references}
\addcontentsline{toc}{section}{References}

\hypertarget{refs}{}
\hypertarget{ref-Einstein1905}{}
Einstein, Albert. 1905. ``On the Electrodynamics of Moving Bodies.'' \emph{Annalen Der Physik} 322 (10): 891--921. doi:\href{https://doi.org/10.1002/andp.19053221004}{10.1002/andp.19053221004}.

\hypertarget{ref-Newton1730}{}
Newton, Isaac. 1730. \emph{Opticks, or a Treatise of the Reflections, Refractions, Inflections and Colours of Light}. William Innys. \url{http://books.google.com/books?id=XXu4AkRVBBoC}.

%%% MARTIN
\subsubsection*{Acknowledgements}
{\small We thank R. Penrose, who time-travelled to Isaac and Albert, and initiated communication.
Thanks also to the anonymous reviewer who greatly improved this manuscript.}
%%%
\end{document}
